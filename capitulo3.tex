\chapter{Configura\c c\~oes avan\c cadas}

Este cap\'itulo tamb\'em concentra recomenda\c c\~oes de seguran\c ca que dizem respeito \`a interface do usu\'ario. Por\'em, devido \`as suas caracter\'isticas peculiares, destinam-se a dispositivos iOS nos quais a seguran\c ca \'e primordial. Estas recomenda\c c\~oes podem impactar significamente a usabilidade do dispositivo, por isso \'e recomendado consider\'a-las como medidas de defesa em profundidade.

\section{Desabilitar o uso de c\'odigos de desbloqueio simples}

Esta recomenda\c c\~ao determina que c\'odigos de bloqueio de apenas quatro d\'igitos n\~ao sejam permitidos para se proteger o acesso ao dispositivo. \'E recomendado que o dispositivo seja configurado para permitir o uso de c\'odigos de bloqueio com mais de quatro caracteres alfan\'umericos (n\'umeros, letras, s\'imbolos, etc).

Permitir uma senha alfanum\'erica para desbloquear o dispositivo iOS aumenta a dificuldade que uma pessoa mal intencionada ter\'a na tentativa de realizar acessos n\~ao autorizados.

\begin{enumerate}
\item Pressionar Ajustes
\item Pressionar C\'odigo, ou Touch ID e C\'odigo
\item Pressionar Alterar C\'odigo, ou Ativar C\'odigo
\item Caso algum c\'odigo j\'a tenha sido configurado, ser\'a necess\'ario digit\'a-lo
\item Pressionar Op\c c\~oes de c\'odigo
\item Pressionar C\'odigo Alfanum\'erico Personalizado
\item Digitar um c\'odigo alfanum\'erico e pressionar Seguinte
\item Digitar novamente o c\'odigo alfanum\'erico e pressionar OK
\end{enumerate}

\section{Habilitar a elimina\c c\~ao de informa\c c\~oes}

Esta configura\c c\~ao determina que o dispositivo iOS apague todo o seu conte\'udo (v\'ideos, fotos, etc) ap\'os dez tentativas fracassadas de desbloqueio.

Sucessivas tentativas fracassadas de desbloqueio do aparelho sugerem que ele n\~ao se encontra nas m\~aos de seu propriet\'ario. Neste caso, apagar todo o conte\'udo garante a confidencialidade das informa\c c\~oes armazenadas no dispositivo.

\begin{enumerate}
\item Pressionar Ajustes
\item Pressionar C\'odigo, ou Touch ID e C\'odigo
\item Pressionar Alterar C\'odigo, ou Ativar C\'odigo
\item Caso algum c\'odigo j\'a tenha sido configurado, ser\'a necess\'ario digit\'a-lo
\item Ativar a op\c c\~ao Eliminar Dados
\end{enumerate}

\section{Desativar o desbloqueio atrav\'es do Touch ID}

O Touch ID permite o uso de uma ou mais impress\~oes digitais como c\'odigo de desbloqueio, atrav\'es de um simples toque do bot\~ao Home. O sensor do Touch ID ``l\^e'' a impress\~ao digital e automaticamente desbloqueia o telefone.

Desabilitar este recurso evita o risco de uma autentica\c c\~ao n\~ao autorizada via Touch ID, seja atrav\'es de falsos positivos, seja atrav\'es de ataques intencionais. 

\begin{enumerate}
\item Pressionar Ajustes
\item Pressionar Geral
\item Pressionar C\'odigo, ou Touch ID e C\'odigo
\item Caso algum c\'odigo j\'a tenha sido configurado, ser\'a necess\'ario digit\'a-lo
\item Desativar a op\c c\~ao Desbloquear iPhone, ou Desbloquear iPad
\end{enumerate}

Vale lembrar que esta configura\c c\~ao se aplica apenas a dispositivos iOS mais recentes.

\section{Desabilitar o acesso \`a Central de Controle a partir da tela bloqueada}

Ao longo da hist\'oria do sistema iOS, j\'a foram descobertas algumas maneiras de se contornar a prote\c c\~ao da tela bloqueada. Esta recomenda\c c\~ao desabilita o acesso \`a Central de Controle a partir da tela. 

A ideia aqui \'e eliminar a possibilidade de a Central de Controle vir ser usada como um  meio para se contornar a prote\c c\~ao da tela bloqueada.

\begin{enumerate}
\item Pressionar Ajustes
\item Pressionar Central de Controle
\item Desativar a op\c c\~ao Acesso na Tela Bloqueada
\end{enumerate}

\section{``Esquecer'' redes Wi-Fi}

Esta configura\c c\~ao faz com que o dispositivo iOS ``esque\c ca'' redes Wi-Fi nas quais ele j\'a tenha conectado. 

Uma rede Wi-Fi confi\'avel, mas sem autentica\c c\~ao, pode ser mascarada e o dispositivo pode se conectar automaticamente a ela se a mesma n\~ao tiver sido ``esquecida'' pelo dispositivo desde a \'ultima conex\~ao.

Outra situa\c c\~ao poss\'ivel \'e quando a rede Wi-Fi mant\'em seu nome padr\~ao, de f\'abrica, e o dispositivo iOS encontra uma outra rede n\~ao confi\'avel de mesmo nome e acabe tentando se conectar a ela automaticamente.

Para esta recomenda\c c\~ao, \'e necess\'ario que o Wi-Fi esteja ativado e a rede Wi-Fi a ser esquecida esteja pr\'oxima ao dispositivo. 

\begin{enumerate}
\item Pressionar Ajustes
\item Pressionar Wi-Fi
\item Na lista ESCOLHA UMA REDE..., localizar a rede a ser esquecida e pressionar o s\'imbolo de exclama\c c\~ao.
\item Na tela seguinte, pressionar Esquecer Esta Rede
\item Confirmar o esquecimento da rede
\end{enumerate}

\section{Desabilitar todo o recurso de redes Wi-Fi}

Em ambientes onde a seguran\c ca \'e prioridade, recomenda-se que o recurso de conex\~ao a redes Wi-Fi permane\c ca desabilitado no dispositivo Android. Caso ele possua acesso a servi\c cos de dados celulares (3G ou 4G por exemplo), a conex\~ao \`a internet dever\'a ocorrer atrav\'es destas redes.

\begin{enumerate}
\item Pressionar Ajustes
\item Pressionar Wi-Fi
\item Desativar a Wi-Fi
\end{enumerate}

\section{Desabilitar o Acesso Pessoal}

O Acesso Pessoal permite que o usu\'ario compartilhe sua conex\~ao \`a internet, via 3G ou 4G, com outros dispositivos, atrav\'es de Wi-Fi, Bluetooth, ou cabo USB. 

Desabilitar o Acesso Pessoal, quando o mesmo n\~ao \'e necess\'ario, elimina a possibilidade de o recurso vir a ser usado como um meio para que pessoas mal intencionadas tecnicamente preparadas ataquem remotamente o dispositivo iOS.

\begin{enumerate}
\item Pressionar Ajustes
\item Pressionar Celular, Dados Celulares
\item Desativar o Acesso Pessoal, caso ele se encontre ativado
\end{enumerate}

\section{Desabilitar o servi\c co de localiza\c c\~ao}

O servi\c co de localiza\c c\~ao permite que alguns aplicativos instalados no dispositivo iOS obtenham e usem informa\c c\~oes que indiquem a localiza\c c\~ao f\'isica do usu\'ario. Esta localiza\c c\~ao \'e determinada atrav\'es do GPS do dispositivo, da rede celular 3G ou 4G, e de redes Wi-Fi. 
Se o usu\'ario desativar os servi\c cos de localiza\c c\~ao, ele receber\'a do solicita\c c\~oes para reativ\'a-la, sempre que algum aplicativo quiser fazer uso deste recurso.

Manter o servi\c co de localiza\c c\~ao ativado aumenta a capacidade de pessoas mal intencionadas, com consider\'avel conhecimento t\'ecnico, rastrearem a localiza\c c\~ao do usu\'ario atrav\'es de sites web, aplicativos, etc.

\begin{enumerate}
\item Pressionar Ajustes
\item Pressionar Privacidade
\item Pressionar Serv. Localiza\c c\~ao
\item Na tela seguinte, desativar o servi\c co de localiza\c c\~ao
\end{enumerate}

\section{Habilitar o Modo Avi\~ao}

Quando est\'a habilitado, o Modo Avi\~ao desativa todos os transmissores e receptores de sinais de r\'adio do dispositivo iOS. Alguns servi\c cos desativados, s\~ao:

\begin{itemize}
\item Envio e recebimento de liga\c c\~oes
\item Envio e recebimento de SMS e MMS
\item Dados m\'oveis (3G, 4G)
\item GPS
\item Wi-Fi
\item Bluetooth
\end{itemize}

Assim, quando estas funcionalidades forem desnecess\'arias, \'e recomendado manter o dispositivo no Modo Avi\~ao. Caso a transmiss\~ao e recep\c c\~ao de sinais permane\c cam habilitados mesmo sem necessidade, haver\'a um aumento da possibilidade de que estes sinais de r\'adio sejam usados como um  meio para se atacar remotamente o dispositivo.

\begin{enumerate}
\item Pressionar Ajustes
\item Ativar o Modo avi\~ao
\end{enumerate}

\section{N\~ao exibir notifica\c c\~oes de aplicativos na tela bloqueada}

Esta configura\c c\~ao previne que notifica\c c\~oes oriundas de aplicativos instalados sejam exibidas quando o dispositivo iOS encontra-se bloqueado. 

\'E recomendado que tal visualiza\c c\~ao seja desabilitada para todos os aplicativos nos quais \'e desejada confidencialidade. Do contr\'ario, pessoas que n\~ao possuem o c\'odigo de desbloqueio poder\~ao visualizar notifica\c c\~oes, aumentando assim o risco de vazamento de informa\c c\~oes importantes.

\begin{enumerate}
\item Pressionar Ajustes
\item Pressionar Notifica\c c\~oes
\item Na lista ESTILO DA NOTIFICA\c C\~AO, localizar o aplicativo e pression\'a-lo
\item Na tela seguinte, desativar a op\c c\~ao Mostrar na Tela Bloqueada
\item Repetir os passos 3 e 4 para outros aplicativos  
\end{enumerate}

\section{Habilitar o recurso Buscar iPhone (ou Buscar iPad)}

Esta configura\c c\~ao habilita as seguintes funcionalidades no dispositivo iOS:

\begin{itemize}
\item o rastreamento remoto da localiza\c c\~ao do dispositivo
\item a elimina\c c\~ao remota de informa\c c\~oes armazenadas no dispositivo
\item a exibi\c c\~ao remota de mensagens
\end{itemize}

Habilitar o recurso Buscar iPhone (ou Buscar iPad) no iOS 9 ativa a capacidade de se localizar o dispositivo atrav\'es do aplicativo iOS Buscar iPhone, ou atrav\'es do site do iCloud. Tamb\'em exibe uma mensagem personalizada com um n\'umero de telefone \`a sua escolha na tela bloqueada, e previne a execu\c c\~ao de a\c c\~oes importantes sem a digita\c c\~ao da senha do Apple ID, como a elimina\c c\~ao das informa\c c\~oes armazenadas no dispositivo iOS e sua configura\c c\~ao.

\begin{enumerate}
\item Pressionar Ajustes
\item Pressionar iCloud
\item Pressionar Buscar iPhone (ou Buscar iPad)
\item Na tela seguinte, ativar a op\c c\~ao Buscar iPhone (ou Buscar iPad)
\end{enumerate}

\section{Habilitar uma senha para acesso ao cart\~ao SIM}

O cart\~ao SIM, conhecido como chip da operadora de telefonia celular, permite a realiza\c c\~ao de liga\c c\~oes telef\^onicas, al\'em do armazenamento de informa\c c\~oes de contatos, e de outras informa\c c\~oes pessoais. Este controle assegura que o cart\~ao SIM, caso seja perdido ou roubado, n\~ao ser\'a usado em nenhum outro dispositivo.

Uma pessoa maliciosa pode retirar o cart\~ao SIM de um telefone e inseri-lo em outro telefone, permitindo o envio de mensagens, a realiza\c c\~ao de liga\c c\~oes, etc. Adicionar uma senha ao cart\~ao proteger\'a contra este tipo de ataque. Tal senha ser\'a requisitada pelo dispositivo sempre que o cart\~ao SIM for acessado.

\begin{enumerate}
\item Pressionar Ajustes
\item Pressionar Telefone
\item Pressionar PIN do SIM
\item Digitar um PIN
\end{enumerate}
