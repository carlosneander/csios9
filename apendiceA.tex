\chapter{Refer\^encias}

Neste cap\'itulo, encontram-se listados alguns sites e livros que foram utilizados como fontes de informa\c c\~ao para o desenvolvimento deste guia. 

Cabe aqui uma ressalva sobre os endere\c cos da internet. Mesmo tomando o m\'aximo cuidado ao inseri-los neste cap\'itulo, \'e natural que, com o tempo, eles sejam desativados ou mudem de endere\c co. Ou seja, alguns dos endere\c cos utilizados correm o risco de se tornarem inv\'alidos, ou ``quebrados''. Para os leitores, sugere-se pesquisar no Google pelos endere\c cos atuais, caso algum link abaixo esteja quebrado. 

A maioria das recomenda\c c\~oes descritas aqui foram baseadas no site de suporte da Apple, que encontra-se dispon\'ivel no seguinte endere\c co:

\vspace{5mm}
http://support.apple.com/
\vspace{5mm}

O recurso Find my iPhone pode ser acessado no seguinte endere\c co:

\vspace{5mm}
http://www.apple.com/icloud/find-my-iphone.html
\vspace{5mm}

Com rela\c c\~ao \`a elimina\c c\~ao de todas as informa\c c\~oes armazenadas no dispositivo iOS, uma boa fonte de consulta \'e o livro iPhone and iOS Forensics: Investigation, Analysis and Mobile Security for Apple iPhone, iPad and iOS Devices. Mais detalhes sobre este livro podem ser encontrados no seguinte endere\c co:

\vspace{5mm}
http://textbooks.elsevier.com/web/product\_details.aspx?isbn=9781597496599
\vspace{5mm}

Com rela\c c\~ao ao recurso para impedir o rastreamento durante a navega\c c\~ao, mais detalhes sobre a iniciativa \textit{Do Not Track} podem ser encontrados nos seguintes sites: 

\begin{itemize}
\item Do Not Track - Universal Web Tracking Opt Out (http://donottrack.us/)
\item W3C Tracking Protection Working Group (http://www.w3.org/2011/tracking-protection/)
\item Do Not Track | Electronic Frontier Foundation (https://www.eff.org/issues/do-not-track)
\end{itemize}
